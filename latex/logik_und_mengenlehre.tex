\section{Logik und Mengenlehre (4-21)}
Definition von logischer Negation, Konjunktion, Disjunktion, Implikation, Äquivalenz. Bestimmung von Wahrheitswerten von Aussageformen. Definition von Gleichheit von Mengen, Teilmenge (Inklusion), Obermenge, Durchschnitt, Vereinigung, Differenz von Mengen, Komplement, Disjunktheit und disjunkte Vereinigung, Potenzmenge. 
\subsection{Definition von logischer Negation, Konjunktion, Disjunktion, Implikation, Äquivalenz. (5f)}
\begin{equation}
\begin{split}
\text{Negation} \qquad & \neg A \\
\text{Konjunktion (and)}  \qquad & A \wedge B \\
\text{Disjunktion (or)}  \qquad & A \vee B \\
\text{Implikation}  \qquad & (A \Rightarrow B) \Leftrightarrow (\neg A \vee B) \Leftrightarrow (\neg B \Rightarrow \neg A) \\
\text{Äquivalenz}  \qquad & (A \Leftrightarrow B) \Leftrightarrow ((A \Rightarrow B) \wedge (B \Rightarrow A))
\end{split}
\end{equation}

\subsection{Bestimmung von Wahrheitswerten von Aussageformen.}

\subsection{Definition von Gleichheit von Mengen, Teilmenge (Inklusion), Obermenge, Durchschnitt, Vereinigung, Differenz von Mengen, Komplement, Disjunktheit und disjunkte Vereinigung, Potenzmenge. (12ff)}
\subsubsection{Gleichheit von Mengen}
$M = N$ wenn alle Elemente von $N$ in $M$ vorkommen. $\Rightarrow$ Wir können nur Aussagen über Mengen treffen wenn wir alle ihre Elemente kennen. $\forall x \in M: x \in N \wedge \forall y \in N: y \in M$
\subsubsection{Teilmenge (Inklusion)}
$M \subseteq N$, wenn alle Elemente aus $M$ in $N$ vorkommen. $\forall x \in M: x \in N$ \\
$M \subset N$, wenn Elemente in $N$ vorhanden sind, die in $M$ nicht vorkommen spricht man von einer echten Teilmenge. $\exists x \in N: x \notin M$

\subsubsection{Obermenge}
Ist $M \subseteq N$, dann ist $N$ die Obermenge von $M$.

Ist $M$ Obermenge von $N$ und umgekehrt, so sind die Mengen äquivalent, $M = N$

\subsubsection{Durschnitt}
$M,N : x \in M \wedge x \in N$
Der Durchschnitt zweier Mengen $M \cap N$ ist die Menge an Elementen, die in beiden Mengen $M$, $N$ vorkommen.

\subsubsection{Vereinigung}
$M,N: x \in M \vee x \in N$
Die Vereinigung zweier Mengen $M \cup N$ ist die Menge an Elementen, die in entweder $M$ oder $N$ vorkommen.

\subsubsection{Differenz von Mengen}
Die Differenz zweier Mengen $M\setminus N$ ist die Menge an Elementen, die in $M$ vorkommen, aber nicht in $N$.
$M,N : x \in M \wedge x \notin N$

\subsubsection{Komplement}
Ist Menge $N$ eine Teilmenge der Menge $M$ ($M \subseteq N$, $M$ ist Universum von $N$), dann wird $M \setminus N$ auch das Komplement $N$ zu $M$ genannt. $N^c := M \setminus N$

\subsubsection{Disjunktheit und disjunkte Vereinigung}
Ist der Durchschnitt zweier Mengen $M \cap N$ die leere Menge, so sind diese Mengen disjunkt. Die Vereinigung zweier disjunkter Mengen lautet dann eine disjunkte Vereinigung und wird $M \dot{\cup} N$ geschrieben.

\subsubsection{Potenzmenge}
Die Potenzmenge einer Menge ist die Menge aller Teilmengen:\\
$M = \{0,1,2\}$, $\mathcal{P}(M)= \{\emptyset, \{0\}, \{1\}, \{2\}, \{0,1\}, \{0,2\}, \{1,2\}, \{0,1,2\}\}$
