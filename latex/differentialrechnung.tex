\section{Differentialrechnung}
 
Definition der Differenzierbarkeit und Ableitung für reell- und für komplexwertige Funktionen. Satz: Eine komplexwertige Funktion ist genau dann differenzierbar, wenn ihr Realteil und ihr Imaginärteil differenzierbar sind. Satz: Differenzierbarkeit impliziert Stetigkeit. Ableitungsregeln: Ableiten ist linear, Produktregel, Quotientenregel, Kettenregel. Ableitungen von Polynomen, von der reellen Exponentialfunktion, vom natürlichen Logarithmus, vom reellen Sinus und vom reellen Kosinus. Ableitung von Potenzfunktionen mit reellen Exponenten. Definition von Ableitungen höherer Ordnung. Satz: Ist eine Funktion in einem lokalen Extremum differenzierbar, so verschwindet dort die Ableitung, Satz über den Zusammenhang des Vorzeichens der Ableitung mit der Monotonie einer differenzierbaren Funktion. Hinreichende Bedingungen für die Existenz von Maxima und Minima bei differenzierbaren Funktionen. 