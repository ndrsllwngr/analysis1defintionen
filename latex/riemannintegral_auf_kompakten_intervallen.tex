\section{Riemannintegral auf kompakten Intervallen (139-162)}
 
Satz: Stetige Funktionen auf kompakten Intervallen sind integrierbar. Linearität des Riemannintegrals. Hauptsatz der Differential- und Integralrechnung. Definition der Stammfunktion. Partielle Integration. Substitutionsformel. Aufgaben vom Typ wie in den Beispielen 10.21, 10.23 und 10.25 lösen können.


\subsection{Linearität des Riemannintegrals (144)}
$\text{Let } a ,b \in \mathbb { R } ,a \leq b ,I : = [ a ,b ]$.

The integral is linear: More precicely if $f ,g \in \mathcal { R } ( I ,\mathbb { K } ) \text{ and } \lambda ,\mu \in \mathbb { K } ,\text{ then } \lambda f + \mu g \in \mathcal { R } ( I ,\mathbb { K } )$ and 
\begin{equation}
\int ( \lambda f + \mu g ) = \lambda \int _ { I } f + \mu \int _ { I } g
\end{equation}

%\begin{figure}[H]
%	\centering
%  \includegraphics[width=0.7\textwidth]{../img/{144-theorem-10-11}.png}
%	\caption{144-theorem-10-11}
%	\label{144-theorem-10-11}
%\end{figure}

\subsection{Hauptsatz der Differential- und Integralrechnung (152)}

\begin{leftbar}
$\text{ If } a ,b \in \mathbb { R } ,a \leq b ,I : = [ a ,b ] ,f : I \rightarrow \mathbb { C }$, then denote

\noindent\begin{minipage}{.5\linewidth}
\begin{equation}
\begin{split}
\int _ { a } ^ { b } f &: = \int _ { I } f, \\ 
[ f ( t ) ] _ { a } ^ { b } &: = [ f ] _ { a } ^ { b } : = f ( b ) - f ( a )
\end{split}
\end{equation}
\end{minipage}%
\begin{minipage}{.5\linewidth}
\begin{equation}
\begin{split}
\int _ { b } ^ { a } f &: = - \int _ { a } ^ { b } f, \\
[ f ( t ) ] _ { b } ^ { a } &: = [ f ] _ { b } ^ { a } : = f ( a ) - f ( b )
\end{split}
\end{equation}
\end{minipage}%

where $f \in \mathcal { R } ( I ,\mathbb { C } )$.

\end{leftbar}
%\begin{figure}[H]
%	\centering
%  \includegraphics[width=0.7\textwidth]{../img/{152-notation-10-18}.png}
%	\caption{152-notation-10-18}
%	\label{152-notation-10-18}
%\end{figure}

$\text{Let } a ,b \in \mathbb { R } ,a < b ,I : = [ a ,b ]$.
\begin{enumerate}
\item $\text{ If } f \in \mathcal { R } ( I ,\mathbb { K } )$ is continuous in $\xi \in I$, then, for each $c \in I$, the function
\begin{equation}
F _ { c } : I \rightarrow \mathbb { K } ,\quad F _ { c } ( x ) : = \int _ { c } ^ { x } f ( t ) d t,
\end{equation}
is \textbf{differentiable} in $\xi$ with $F _ { c } ^ { \prime } ( \xi ) = f ( \xi )$. In particular, if $f \in C ( I ,\mathbb { K } )$, then $F _ { c } \in C ^ { 1} ( I ,\mathbb { K } )$ and $F _ { c } ^ { \prime } ( x ) = f ( x )$ for each $x \in I$.

\item If $F \in C ^ { 1} ( I ,\mathbb { K } )$ or, alternatively, $F : I \rightarrow \mathbb { K }$ is differentiable with integrable derivative $F ^ { \prime } \in \mathcal { R } ( I ,\mathbb { K } )$, then 
\begin{equation}
F ( b ) - F ( a ) = [ F ( t ) ] _ { a } ^ { b } = \int _ { a } ^ { b } F ^ { \prime } ( t ) \text{d} t,
\end{equation}
and
\begin{equation}
F ( x ) = F ( c ) + \int _ { c } ^ { x } F ^ { \prime } ( t ) d t \quad \text{ for each } c ,x \in I.
\end{equation}

\end{enumerate}


%\begin{figure}[H]
%	\centering
%  \includegraphics[width=0.7\textwidth]{../img/{152-theorem-10-19}.png}
%	\caption{152-theorem-10-19}
%	\label{152-theorem-10-19}
%\end{figure}

\subsection{Definition der Stammfunktion (153)}

$\text{ If } I \subseteq \mathbb { R } ,f : I \rightarrow \mathbb { K } ,\text{ and } F : I \rightarrow \mathbb { R }$ is a differentiable function with $F ^ { \prime } = f$, then $F$ is called a \textit{primitive} or \textit{antiderivative }of $f$.

%\begin{figure}[H]
%	\centering
%  \includegraphics[width=0.7\textwidth]{../img/{153-definition-10-20}.png}
%	\caption{153-definition-10-20}
%	\label{153-definition-10-20}
%\end{figure}

\subsection{Partielle Integration (154)}

\begin{figure}[H]
	\centering
  \includegraphics[width=0.7\textwidth]{../img/{154-theorem-10-22}.png}
	\caption{154-theorem-10-22}
	\label{154-theorem-10-22}
\end{figure}

\subsection{Substitutionsformel (154)}

\begin{figure}[H]
	\centering
  \includegraphics[width=0.7\textwidth]{../img/{154-theorem-10-24}.png}
	\caption{154-theorem-10-24}
	\label{154-theorem-10-24}
\end{figure}

\subsection{Aufgaben vom Typ wie in den Beispielen 10.21, 10.23 und 10.25}

\begin{figure}[H]
	\centering
  \includegraphics[width=0.7\textwidth]{../img/{153-example-10-21}.png}
	\caption{153-example-10-21}
	\label{153-example-10-21}
\end{figure}

\begin{figure}[H]
	\centering
  \includegraphics[width=0.7\textwidth]{../img/{154-example-10-23}.png}
	\caption{154-example-10-23}
	\label{154-example-10-23}
\end{figure}

\begin{figure}[H]
	\centering
  \includegraphics[width=0.7\textwidth]{../img/{155-example-10-25}.png}
	\caption{155-example-10-25}
	\label{155-example-10-25}
\end{figure}