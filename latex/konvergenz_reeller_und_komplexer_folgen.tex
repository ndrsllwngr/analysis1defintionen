\section{Konvergenz reeller und komplexer Folgen}
 
Definition und Veranschaulichung der Epsilon-Umgebung sowie der Umgebung einer reellen Zahl sowie einer komplexen Zahl. Definition der Beschränktheit reeller und komplexer Folgen. Definition von Konvergenz und Grenzwert/Limes reeller und komplexer Folgen. Definition von Divergenz reeller und komplexer Folgen. Satz: Konvergente Folgen sind beschränkt. Nullfolgen sind solche, die gegen Null konvergieren. Satz: Eine durch eine Nullfolge beschränkte Folge ist selbst eine Nullfolge. Satz: Das Produkt aus einer Nullfolge und einer beschränkten Folge ist eine Nullfolge. Die Grenzwertsätze aus Th. 7.13 wissen und zur Bestimmung von Grenzwerten anwenden können. Einschachtelungssatz. Definition der bestimmten Divergenz gegen plus oder minus Unendlich. Eine monoton steigende Folge konvergiert oder divergiert bestimmt gegen plus Unendlich; eine monoton fallende Folge konvergiert oder divergiert bestimmt gegen minus Unendlich. Definition von Teilfolge und Umordnung einer Folge. Satz: Jede Teilfolge und jede Umordnung einer konvergenten Folge ist konvergent mit dem selben Limes. 